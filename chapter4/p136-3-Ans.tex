\begin{Exercise}
  证明:单调函数的不连续点的集合至多为可数集。
\end{Exercise}

首先,让我们回顾一下刻画实数系的公理系统。

     设$x, y, z \in \mathbb{R}$,则
   \begin{enumerate}
   \item   $x + y = y + x$
   \item   $(x + y) + z = x + (y + z)$
   \item   $0 + x = x + 0 = x$
   \item   $(-x) + x = x + (-x) = 0$
   \item   $x * y = y * x$
   \item   $(x * y) * z = x * (y *z)$
   \item   $1 * x = x * 1 = x$
   \item   $\forall x \in \mathbb{R} x \neq 0 \to x^{-1} * x = x * x^{-1} = 1$
   \item   $x* (y + z) = x * y + x * z$
   \item   $(y + z) * x = y * x + z * x$
       \item $x \leq x$
   \item $ x \leq y \land y \leq x \rightarrow x = y$
   \item $x \leq y \land y \leq z \rightarrow x \leq z$
   \item $x \leq y \lor y \leq x$ 
\item $x > y \rightarrow x + z > y + z$
\item $x > y \land z >0 \rightarrow x * z > y * z$
\item   $\forall A \subseteq \mathbb{R} (A \neq \phi \land \exists x \in \mathbb{R} (\forall y \in A (y \leq x)) \rightarrow \exists z \in R ((\forall y \in A (y \leq z) )\land ( \forall x \in \mathbb{R} (\forall y \in A (y \leq x) \rightarrow z \leq x))))$   
    \end{enumerate}

其中最后一条公理为:实数集的任意一个非空子集如果有上界,则必有上确界。

接下来,我们来看结论中涉及的一些基本概念。

设$f:R\to R$为一个函数。如果对任意的$x_1\in R$,$x_2\in R$,$x_1< x_2$,那么$f(x_1) \leq f(x_2)$,则称$f$为单调函数。

设$x_0\in R$,$L\in R$,如果对任意的$\varepsilon\in R$, $\varepsilon> 0$,存在$\delta \in R$, $\delta > 0$,只要$|x-x_0|<\delta$,就有$|f(x) - L|<\varepsilon$,则称函数$f$在$x_0$处的极限为$L$,记为$\lim_{x\to x_0}f(x)=L$。

如果$\lim_{x\to x_0}f(x)=f(x_0)$,则称函数$f$在$x_0$处连续,$x_0$为函数$f$的连续点;如果函数$f$在$x_0$处不连续,则称$x_0$为函数$f$的不连续点。

\begin{proof}[证明]
  对任意的$x_0\in R$,由单调函数的定义知,集合$\{f(x)|x<x_0\}$有上界$f(x_0)$,从而有上确界,定义$L(x_0)=sup \{f(x)|x<x_0\}$;集合$\{f(x)|x>x_0\}$有下界$f(x_0)$,从而有下确界,定义$U(x_0)=inf \{f(x)|x>x_0\}$。如果$x_1<x_2$,那么$U(x_1)\leq f(\frac{x_1+x_2}{2}) \leq L(x_2)$。另外,如果$x_0$为$f$的不连续点,可以证明$L(x_0) <  U(x_0)$。因此,集合$S=\{(L(x),U(x))|x\text{为函数}$f$\text{的不连续点}\}$中的开区间两两不相交。在$S$中的每个开区间中取一个有理数,则所有这些有理数的集合与函数$f$的所有不连续点构成的集合是对等的,从而$f$的所有不连续点所构成的集合为至多可数的。

  设$x_0$为$f$的不连续点,以下证明$L(x_0) <  U(x_0)$。由$L(x_0)\leq f(x_0) \leq U(x_0)$知$L(x_0) \leq U(x_0)$,因此只需证$L(x_0)\neq U(x_0)$。用反证法,假设$L(x_0)=U(x_0)$,则$L(x_0)=U(x_0)=f(x_0)$。对任意的$\varepsilon >0$,由$L(x_0)$的定义知存在$x'<x_0$使得$f(x')>L(x_0)-\varepsilon=f(x_0)-\varepsilon$;由$U(x_0)$的定义知存在$x''>x_0$使得$f(x'')<U(x_0) + \varepsilon=f(x_0) + \varepsilon$。设$\delta = \min (|x'-x_0|, |x''-x_0|)$,那么当$|x-x_0|< \delta$时,就有$|f(x)-f(x_0)|<\varepsilon$,从而$\lim_{x\to x_0}=f(x_0)$,函数$f$在$x_0$处连续,这与$x_0$为$f$的不连续点矛盾。
\end{proof}
%%% Local Variables:
%%% mode: latex
%%% TeX-master: "p136-3"
%%% End:
