\documentclass{article}
\usepackage{tikz}
\usepackage{CJKutf8}
\usepackage{amsmath}
\usepackage{amsthm}
\begin{document}
\begin{CJK}{UTF8}{gbsn}
\newtheorem*{Exercise}{习题}
\begin{Exercise}
  设$R$,$S$为集合$X$上的等价关系。如果$R\circ S$为等价关系,则$(R\cup S)^+\subseteq R\circ S$。
\end{Exercise}
\begin{proof}[证明]
  对任意的$a\in X,c\in X$,由$(a,c)\in (R\cup S)^+$,往证$(a,c)\in R\circ S$。

  对任意的$a\in X,c\in X$,如果$(a,c)\in (R\cup S)^+$,则存在自然数$n$,$n\geq 1$, $(a,c)\in (R\cup S)^n$。

  以下用数学归纳法证明,对任意的自然数$n$,$n\geq 1$,$(R\cup S)^n\subseteq R\circ S$。

  (1)当$n=1$时,对任意的$a\in X,c\in X$,如果$(a,c)\in R\cup S$,则$(a,c)\in R$或者$(a,c)\in S$。如果$(a,c)\in R$,此时由$S$为等价关系知$(c,c)\in S$,从而$(a,c)\in R\circ S$; 如果$(a,c)\in S$,此时由$R$为等价关系知$(a,a)\in R$,从而$(a,c)\in R\circ S$。

  (2)假设当$n=k(k\geq 1)$时结论成立,往证当$n=k+1$时结论也成立。

  由$R$,$S$,$R\circ S$都为$X$上的等价关系知,$S\circ R=S^{-1}\circ R^{-1}=(R\circ S)^{-1}=R\circ S$。

  对任意的$a\in X,c\in X$,如果$(a,c)\in (R\cup S)^{k+1}=(R\cup S)^k\circ (R\cup S)$,则存在$b\in X$,$(a,b)\in (R\cup S)^k$并且$(b,c)\in (R\cup S)$。由归纳假设,$(a,b)\in R\circ S$。如果$(b,c)\in R$,那么$(a,c)\in (R\circ S)\circ R = R\circ (S\circ R) = R\circ (R\circ S) = (R\circ R)\circ S = R^2\circ S \subseteq R\circ S$;如果$(b,c)\in S$,那么$(a,c)\in (R\circ S)\circ S = R\circ (S\circ S) = R\circ S^2 \subseteq R\circ S$。
  

  
\end{proof}


\end{CJK}
\end{document}

%%% Local Variables:
%%% mode: latex
%%% TeX-master: t
%%% End:

