\documentclass{book}[oneside]
\usepackage{CJKutf8}
\usepackage{amsmath}
\usepackage{amsfonts}
\usepackage{amsthm}
\usepackage{titlesec}
\usepackage{titletoc}
\usepackage{xCJKnumb}


\usepackage{tikz}
\titleformat{\chapter}{\centering\Huge\bfseries}{第\, \xCJKnumber{\thechapter}\,
    章}{1em}{}
  % \renewcommand{\chaptermark}[1]{\markboth{第 \thechapter 章}{}}
\usepackage{mathrsfs}

\newtheorem{Def}{定义}[chapter]
\newtheorem{Thm}{定理}[chapter]
\newtheorem{Cor}{推论}[chapter]
\newtheorem{Ax}{公理}[chapter]

\newtheorem{Exercise}{练习}[chapter]

\newtheorem{Example}{例}[chapter]


\begin{document}
\begin{CJK*}{UTF8}{gbsn}
  \title{离散数学讲义}
  \author{陈建文}
  \maketitle
  % \tableofcontents
  
  \underline{课程学习目标:}
\begin{enumerate}
\item 训练自己的逻辑思维能力和抽象思维能力
\item 训练自己利用数学语言准确描述计算机科学问题和电子信息科学问题的能力
\end{enumerate}

\underline{学习方法:}
\begin{enumerate}
\item MOOC自学
\item 阅读该讲义
\item 做习题
\item 学习过程中有不懂的问题,在课程QQ群中与老师交流
\end{enumerate}

\underline{授课教师QQ:}2129002650


  \setcounter{chapter}{6}
  \chapter{树}

\begin{Def}
    连通且无圈的无向图称为无向树,简称{\bfseries树}。 一个没有圈的无向图
    称为无向森林,简称{\bfseries森林}。
  \end{Def}
  \begin{Thm}
  设$G=(V,E)$为一个$(p,q)$图,下列各命题等价:
  \begin{enumerate}
  \item $G$为树;
  \item $G$的任意两个不同的顶点间有唯一的一条路联结;
  \item $G$为连通的且去掉任意一条边则得到一个不连通的图;
  \item $G$为连通的且$q = p - 1$;
  \item $G$中无圈且$q = p - 1$;
  \item $G$中无圈且$G$中任意两个不邻接的顶点间加一条边则得到一个含有圈的图。
  \end{enumerate}
  \end{Thm}
  \begin{proof}[证明]

    $1\Rightarrow2$

    用反证法。假设图$G$中存在两个顶点$u$和$v$,在它们之间存在两条不同的路$P_1$和
    $P_2$。由于$P_1\neq P_2$,$P_1$上存在一条边$x=u_1v_1$不在$P_2$上。由$P_1$和
    $P_2$上所有的顶点和边构成的$G$的子图记为$P_1\cup P_2$, 则$(P_1\cup P_2)- x$
    是连通的。于是,$(P_1\cup P_2)-x$中存在一条$u_1-v_1$路$P$,$P+x$为$G$的一个
    圈,矛盾。

    $2\Rightarrow3$

显然,图G为连通的。设$uv$为图$G$的任意一条联结顶点$u$和$v$的边,则$uv$为联结顶点
$u$和$v$的唯一的一条路,从图$G$中去掉边$uv$之后,顶点$u$和顶点$v$之间没有路,于
是得到了一个不连通的图。

$3\Rightarrow 4$

用数学归纳法证明,施归纳于顶点数$p$。

当$p=1$时,结论显然成立。

假设当$p=k$时结论成立,往证当$p=k+1$时结论也成立。
由图$G$为连通的且去掉任意一条边则得到一个不连通的图知图$G$中一定存在一个度为1的
顶点$v$。在图$G$中去掉顶点$v$及其与之关联的边,得到图$G'$。则图$G'$为连通的且去
掉任意一条边会得到一个不连通的图,由归纳假设,图$G'$中有$k-1$条边,于是图$G$中有
$k$条边,$q=p-1$成立,定理得证。

$4\Rightarrow 5$

用反证法。假设图$G$中有圈,则去掉圈上的一条边,得到的图仍然为连通的。如果新得到
的图仍然有圈,在圈上再去掉一条边,又会得到一个新的连通的图。如此继续下去,最终会
得到一个连通的没有圈的图。由从$1$到$4$的证明知最后到的图中有$p-1$条边,这与去掉
边之前图$G$中的边数$q=p-1$矛盾。

$5\Rightarrow 6$

设图$G$有$k$个支,则图$G$中的每个支连通且没有圈。设第$i$个支中含有$p_i$个顶点,
$q_i$条边。由$1$到$4$的证明知在第$i$个支中$q_i=p_i-1$。将所有支的边数和顶点数相
加,可得$q = p-k$。于是$k=1$,从而$G$为连通的。设$u$与$v$为图$G$的任意两个不
邻接的顶点,则$u$与$v$之间存在一条路,再在$u$与$v$之间加一条边,则得到一个圈。

$6\Rightarrow 1$

设$u$和$v$为图$G$的任意两个顶点。如果$u$和$v$邻接,则$u$和$v$之间有一条路。如果
$u$和$v$之间不邻接,则在$u$和$v$之间加一条边,会得到一个圈。在该圈上将边$uv$去掉,
则得到$u$与$v$之间的一条路。这证明了$G$为连通的。
\end{proof}
  \begin{Def}
    设$G=(V,E)$为一个图,$G$的一个生成子图$T=(V,F)$如果是树,则称$T$为$G$的{\bfseries 生成树}。
  \end{Def}
  \begin{Thm}
    图$G$有生成树的充分必要条件是$G$为一个连通图。
  \end{Thm}
  \begin{Def}
    设$v$为图$G$的一个顶点,如果$G-v$的支数大于$G$的支数,则称顶点$v$为图$G$的一个{\bfseries 割点}。
  \end{Def}
  \centering
    \begin{tikzpicture}[auto,
    specification/.style ={circle, draw, thick}]
   \node[specification] (A) [label=90:$v_1$] at (1,1)  {};
   \node[specification] (B) [label=90:$v_2$] at (2,0)  {};
   \node[specification] (C) [label=-90:$v_3$] at (1,-1)  {};
   \node[specification] (D) [label=180:$v_4$] at (0,0)  {};
   \node[specification] (E) [label=90:$v_5$] at (4,0) {};
   \node[specification] (F) [label=90:$v_6$] at (5,1) {};
   \node[specification] (G) [label=-90:$v_7$] at (5,-1) {};
   \draw[thick] (A) to  (B);
   \draw[thick] (B) to  (C);
   \draw[thick] (C) to  (D);
   \draw[thick] (D) to  (A);
   \draw[thick] (A) to  (C);
   \draw[thick] (B) to  (D);   
   \draw[thick] (B) to  (E);
   \draw[thick] (E) to  (F);
   \draw[thick] (F) to  (G);
   \draw[thick] (G) to  (E);
 \end{tikzpicture}  
  \begin{Thm}
    设$v$为连通图$G=(V,E)$的一个割点,则下列命题等价:
    \begin{enumerate}
    \item $v$为图$G$的一个割点;
    \item 集合$V\setminus \{v\}$有一个二划分$\{U,W\}$, 使得对任意的$u \in U$,$w \in W$,$v$在联结$u$和$w$的每条路上;
    \item 存在与$v$不同的两个顶点$u$和$w$,使得$v$在每一条$u$与$w$间的路上。
    \end{enumerate}
  \end{Thm}
  \centering
    \begin{tikzpicture}[auto,
    specification/.style ={circle, draw, thick}]
   \node[specification] (A) [label=90:$v_1$] at (1,1)  {};
   \node[specification] (B) [label=90:$v_2$] at (2,0)  {};
   \node[specification] (C) [label=-90:$v_3$] at (1,-1)  {};
   \node[specification] (D) [label=180:$v_4$] at (0,0)  {};
   \node[specification] (E) [label=90:$v_5$] at (4,0) {};
   \node[specification] (F) [label=90:$v_6$] at (5,1) {};
   \node[specification] (G) [label=-90:$v_7$] at (5,-1) {};
   \draw[thick] (A) to  (B);
   \draw[thick] (B) to  (C);
   \draw[thick] (C) to  (D);
   \draw[thick] (D) to  (A);
   \draw[thick] (A) to  (C);
   \draw[thick] (B) to  (D);   
   \draw[thick] (B) to  (E);
   \draw[thick] (E) to  (F);
   \draw[thick] (F) to  (G);
   \draw[thick] (G) to  (E);
 \end{tikzpicture}  
  \begin{Def}
   图$G$的一条边$x$称为$G$的一座{\bfseries 桥},如果$G-x$的支数大于$G$的支数。
  \end{Def}
  \centering
    \begin{tikzpicture}[auto,
    specification/.style ={circle, draw, thick}]
   \node[specification] (A) [label=90:$v_1$] at (1,1)  {};
   \node[specification] (B) [label=90:$v_2$] at (2,0)  {};
   \node[specification] (C) [label=-90:$v_3$] at (1,-1)  {};
   \node[specification] (D) [label=180:$v_4$] at (0,0)  {};
   \node[specification] (E) [label=90:$v_5$] at (4,0) {};
   \node[specification] (F) [label=90:$v_6$] at (5,1) {};
   \node[specification] (G) [label=-90:$v_7$] at (5,-1) {};
   \draw[thick] (A) to  (B);
   \draw[thick] (B) to  (C);
   \draw[thick] (C) to  (D);
   \draw[thick] (D) to  (A);
   \draw[thick] (A) to  (C);
   \draw[thick] (B) to  (D);   
   \draw[thick] (B) to  (E);
   \draw[thick] (E) to  (F);
   \draw[thick] (F) to  (G);
   \draw[thick] (G) to  (E);
 \end{tikzpicture}  

   \begin{Thm}
    设$x$为连通图$G=(V,E)$的一条边,则下列命题等价:
    \begin{enumerate}
    \item $x$为$G$的桥;
    \item $x$不在$G$的任一圈上;
    \item 存在$V$的一个划分$\{U,W\}$,使得对任意的$u \in U, w \in W$,$x$在每一条联结$u$与$w$的路上;
    \item 存在$G$的不同顶点$u$和$v$,使得边$x$在联结$u$和$v$的每条路上。
    \end{enumerate}
  \end{Thm}
  \centering
    \begin{tikzpicture}[auto,
    specification/.style ={circle, draw, thick}]
   \node[specification] (A) [label=90:$v_1$] at (1,1)  {};
   \node[specification] (B) [label=90:$v_2$] at (2,0)  {};
   \node[specification] (C) [label=-90:$v_3$] at (1,-1)  {};
   \node[specification] (D) [label=180:$v_4$] at (0,0)  {};
   \node[specification] (E) [label=90:$v_5$] at (4,0) {};
   \node[specification] (F) [label=90:$v_6$] at (5,1) {};
   \node[specification] (G) [label=-90:$v_7$] at (5,-1) {};
   \draw[thick] (A) to  (B);
   \draw[thick] (B) to  (C);
   \draw[thick] (C) to  (D);
   \draw[thick] (D) to  (A);
   \draw[thick] (A) to  (C);
   \draw[thick] (B) to  (D);   
   \draw[thick] (B) to  (E);
   \draw[thick] (E) to  (F);
   \draw[thick] (F) to  (G);
   \draw[thick] (G) to  (E);
 \end{tikzpicture}  
  \begin{Def}
    设$G = (V,E)$为图,$S \subseteq E$。如果从$G$中去掉$S$中的所有边得到的图$G-S$的支数大于$G$的支数,而去掉$S$的任一真子集中的边得到的图的支数不大于$G$的支数,则称$S$为$G$的一个{\bfseries 割集}。
  \end{Def}
  \centering
    \begin{tikzpicture}[auto,
    specification/.style ={circle, draw, thick}]
   \node[specification] (A) [label=90:$v_1$] at (1,1)  {};
   \node[specification] (B) [label=90:$v_2$] at (2,0)  {};
   \node[specification] (C) [label=-90:$v_3$] at (1,-1)  {};
   \node[specification] (D) [label=180:$v_4$] at (0,0)  {};
   \node[specification] (E) [label=90:$v_5$] at (4,0) {};
   \node[specification] (F) [label=90:$v_6$] at (5,1) {};
   \node[specification] (G) [label=-90:$v_7$] at (5,-1) {};
   \draw[thick] (A) to  (B);
   \draw[thick] (B) to  (C);
   \draw[thick] (C) to  (D);
   \draw[thick] (D) to  (A);
   \draw[thick] (A) to  (C);
   \draw[thick] (B) to  (D);   
   \draw[thick] (B) to  (E);
   \draw[thick] (E) to  (F);
   \draw[thick] (F) to  (G);
   \draw[thick] (G) to  (E);
 \end{tikzpicture}  

\chapter{}
%%% Local Variables:
%%% mode: latex
%%% TeX-master: "book_chapter7"
%%% End:

\end{CJK*}
\end{document}





%%% Local Variables:
%%% mode: latex
%%% TeX-master: t
%%% End:



