\documentclass{article}
\usepackage{tikz}
\usepackage{CJKutf8}
\usepackage{amsmath}
\usepackage{amsthm}
\begin{document}
\begin{CJK}{UTF8}{gbsn}
\newtheorem*{Ex}{习题}
\begin{Ex}
  设$f:X\to Y$。
  
  (1)如果存在唯一的一个映射$g:Y\to X$,使得$gf=I_X$,那么$f$是否可逆呢?

  (2)如果存在唯一的一个映射$g:Y\to X$,使得$fg=I_Y$,那么$f$是否可逆呢?
\end{Ex}
\begin{proof}[解]

  (1)$f$不一定可逆,举例如下:

  设集合$X=\{1\}$,$Y=\{1,2\}$,$f:X\to Y$,$f(1)=1$。则存在唯一的一个映射$g:Y\to X$,$g(1)=1,g(2)=1$,使得$gf=I_X$,但$f$不可逆。


  (2)$f$一定可逆,证明如下:

  由$fg=I_Y$知$f$为满射。以下证明$f$为单射。用反证法,假设$f$不为单射,则存在$x_1\in X$,$x_2\in X$,$x_1\neq x_2$但$f(x_1)=f(x_2)$。

  设$y_0=f(x_1)=f(x_2)$,则$g(y_0)\neq x_1$或者$g(y_0)\neq x_2$。

  令$h:Y\to X$,
  \[h(y)=\begin{cases}
      g(y) &\text{如果} y \neq y_0,\\
      x_1 &\text{如果} y = y_0\text{且} g(y_0) \neq x_1\\
      x_2 &\text{如果} y= y_0 \text{且} g(y_0) \neq x_2
    \end{cases}
  \]
  则$fh=I_Y$,且$h\neq g$,与存在唯一的一个映射$g:Y\to X$,使得$fg=I_Y$,矛盾。
\end{proof}


\end{CJK}
\end{document}


%%% Local Variables:
%%% mode: latex
%%% TeX-master: t
%%% End:
