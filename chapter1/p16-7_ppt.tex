\documentclass{beamer}
\usepackage{ragged2e}
\usepackage{CJKutf8}
\usepackage{tikz}
\setbeamertemplate{theorems}[numbered]
\justifying\let\raggedright\justifying
\begin{document}
\begin{CJK*}{UTF8}{gbsn}


  
\theoremstyle{definition}
\newtheorem{Def}{定义}
\theoremstyle{example}
\newtheorem*{Ex}{例:}
\newtheorem*{Exercise}{习题}

\date{}
\author{陈建文}
\title{习题讲解}

\begin{frame}
  \titlepage
\end{frame}
\begin{frame}

\begin{Exercise}
      设$A,B,C$都是集合,若$A\cup B = A\cup C$且$A\cap B = A\cap C$,试证$B=C$。
\end{Exercise}
\pause
\begin{proof}[证法一]\justifying\let\raggedright\justifying

  先证$B\subseteq C$。

  对任意的$x \in B$,分两种情况讨论:

  1)若$x \in A$:此时$x \in A\cap B$,由$A\cap B = A\cap C$知$x \in A\cap C$,从而$x \in C$。

  2)若$x \notin A$:此时由$x\in B$知$x\in A\cup B$,再由$A\cup B = A\cup C$知$x \in A\cup C$,再由$x \notin A$知$x \in C$。

  综合以上两种情况知对任意的$x$,当$x\in B$时$x\in C$,即$B \subseteq C$。
  

  由$B$和$C$的对称性知$C \subseteq B$,因此$B=C$。
\end{proof}

\end{frame}
\begin{frame}
\begin{Exercise}
      设$A,B,C$都是集合,若$A\cup B = A\cup C$且$A\cap B = A\cap C$,试证$B=C$。
\end{Exercise}
\pause
\begin{proof}[证法二]\justifying\let\raggedright\justifying

  $B= B\cap (A\cup B) = B \cap (A \cup C) = (B \cap A) \cup (B \cap C) = (A \cap B) \cup (B \cap C) = (A \cap C) \cup (B \cap C) = (C \cap A) \cup (C \cap B) = C \cap (A \cup B) = C \cap (A \cup C) = C$
  
\end{proof}  
\end{frame}

\begin{frame}
\begin{Exercise}
      设$A,B,C$都是集合,若$A\cup B = A\cup C$且$A\cap B = A\cap C$,试证$B=C$。
    \end{Exercise}
    \pause
\begin{proof}[证法三]\justifying\let\raggedright\justifying
由已知条件知$(A\cup B)\setminus (A\cap B) = (A\cup C)\setminus (A\cap C)$,从而$A\bigtriangleup B = A \bigtriangleup C$,于是$A \bigtriangleup (A\bigtriangleup B) =A \bigtriangleup (A \bigtriangleup C)$, 由对称差运算的结合律知$(A \bigtriangleup A)\bigtriangleup B =(A \bigtriangleup A) \bigtriangleup C$,即$\phi \bigtriangleup B = \phi \bigtriangleup C$,从而$B = C$。
\end{proof}
  
\end{frame}

\end{CJK*}
\end{document}
