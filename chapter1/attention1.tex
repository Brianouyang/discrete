\documentclass{article}
\usepackage{tikz}
\usepackage{CJKutf8}
\usepackage{amsmath}
\usepackage{amsthm}
\begin{document}
\begin{CJK}{UTF8}{gbsn}
  \title{关于课程PPT}
\maketitle
  
\begin{itemize}
\item 课程PPT下载地址:https://github.com/jianwenchen/discrete
\item chapter1文件夹中chapter1.pdf为ppt, book\_chapter1.pdf为对应ppt的文字版。后面章节依次类推,第一章已经做好,后面章节我会按照课程进度进行完善
\end{itemize}



我们这门课程的学习目标为提高大家的逻辑思维能力和抽象思维能力,因此,大家在阅读PPT时,

\begin{itemize}
\item 每阅读一个概念,你都问问自己这个概念理解了吗?
\item 每阅读一个定理,你都问问自己会证明吗?应该怎么证明?如果不会,再打开MOOC授课视频看看老师是怎么证明的。如果还有不理解的地方,在QQ群里向老师提问,或者加老师QQ单独交流。
\end{itemize}

良好的开端,是成功的一半!第一章的许多概念我们虽然以前已经学过,大家也要从头认真学一遍。因为我们以前在学数学的过程中,有些问题是没有完全讲清楚的,例如,$a^{b+c}=a^b\cdot a^c$,$b$和$c$为整数时你一定会证明,但$b$和$c$为实数时,你会证明吗?在本门课程中,许多概念虽然很抽象,但我们的每个定理都是可以从头开始讲清楚的,这一点对提高我们的逻辑思维能力和抽象思维能力是很重要的。如果只是背诵定理,而不是花时间去思考定理是怎么证明的,这对于提高我们的逻辑思维能力是没有多大帮助的。以前,我们只需要记住$S=\pi r^2$,知道当$r=2$时,$S=4\pi$就可以考个好成绩,或许我们没有太多的去关心为什么$S=\pi r^2$。在本课程中,按照这种思路进行学习,注定是不能取得一个好成绩的。相反,如果我们每遇到一个定理,都能问问自己会证明吗?应该怎样证明?哪怕我们只是理解了本课程一半的定理的证明,对于提升我们的逻辑思维能力也是会有很大帮助的。





\end{CJK}
\end{document}

