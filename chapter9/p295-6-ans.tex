\documentclass{article}
\usepackage{tikz}
\usepackage{CJKutf8}
\usepackage{amsmath}
\usepackage{amsthm}
\begin{document}
\begin{CJK}{UTF8}{gbsn}
  \newtheorem*{Exercise}{习题}
\begin{Exercise}[295-6]
  证明:如果图$G$的任意两个奇数长的圈都有一个公共顶点,则$\chi (G) \leq 5$。
\end{Exercise}
\begin{proof}[证明]
  用反证法,假设$\chi(G)=n$, $n \geq 6$。对图$G$的顶点用$n$种颜色进行着色,使得任意两个相邻的顶点着不同的颜色。

  设$V_1$,$V_2$,$V_3$为其中着3种不同颜色的顶点集合,$V_4$,$V_5$,$V_6$为其中着另外3种不同颜色的顶点集合。则由$V_1\cup V_2\cup V_3$导出的子图$G_1$不是2-可着色的,从而$G_1$中存在一个奇数长的圈$C_1$;同理,由$V_3\cup V_4\cup V_6$导出的子图$G_2$中存在一个奇数长的圈$C_2$。$C_1$和$C_2$没有公共顶点,矛盾。
\end{proof}
\end{CJK}
\end{document}


%%% Local Variables:
%%% mode: latex
%%% TeX-master: t
%%% End:
