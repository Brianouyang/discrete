\documentclass{article}
\usepackage{tikz}
\usepackage{CJKutf8}
\usepackage{amsmath}
\usepackage{amsthm}
\begin{document}
\begin{CJK}{UTF8}{gbsn}
  \newtheorem*{Exercise}{习题}

\begin{Exercise}[294-2]
  设$G$为一个没有三角形的可平面图。应用数学归纳法证明$G$为4-可着色的。
\end{Exercise}
    \begin{proof}[证明]用数学归纳法证明,施归纳于顶点数$p$。

    (1)当$p=1$时,结论显然成立。

    (2)假设当$p=k(k\geq 1)$时结论成立,往证当$p=k+1$时结论也成立。设$G$为包含$k+1$个顶点,没有三角形的可平面图知,$G$中存在一个顶点$v$,$\deg v \leq 3$。显然,$G-v$为包含$k$个顶点,没有三角形的可平面图,由归纳假设,$G-v$为$4$可着色的。假设已经用至多$4$种颜色对$G-v$进行了顶点着色,使得任意相邻的顶点着不同的颜色,那么此时在$G$中与$v$邻接的顶点用了至多$3$种颜色,用另外一种不同的颜色对顶点$v$进行着色,从而用至多$4$种颜色就可以对$G$的顶点进行着色使得相邻的顶点着不同的颜色,即$G$为$4$可着色的。
    
  \end{proof}
\end{CJK}
\end{document}


%%% Local Variables:
%%% mode: latex
%%% TeX-master: t
%%% End:
