\documentclass{article}
\usepackage{tikz}
\usepackage{CJKutf8}
\usepackage{amsmath}
\usepackage{amsthm}
\begin{document}
\begin{CJK}{UTF8}{gbsn}
\newtheorem*{Exercise}{习题}
\begin{Exercise}[p281-3]
  如果$G$为顶点数$p>11$的可平面图,试证$G^c$不是可平面图。
\end{Exercise}
\begin{proof}[证明]  用反证法,假设$G^c$也是可平面图。设$G$有$q$条边,由$G$为可平面图知
  \[q \leq 3p - 6\]
  设$G^c$有$q_1$条边,由$G^c$为有$p$个顶点的可平面图知
  \[q_1 \leq 3p - 6\]
  于是
  \[q + q _1 \leq 6p-12\]
  即
  \[\frac{p(p-1)}{2} \leq 6p-12\]
  \[p^2-p \leq 12p-24\]
  \[p^2-13p+24 \leq 0\]
  当$p\geq 11$时,
  \begin{equation*}
    \begin{split}
     &p^2-13p+24\\
     =&(p-\frac{13}{2})^2-\frac{169}{4}+24\\
     \geq&(11-\frac{13}{2})^2-\frac{169}{4} + 24\\
     =&\frac{81}{4} - \frac{169}{4} + 24\\
     =&2 \geq 0
    \end{split}
  \end{equation*}
  矛盾。
\end{proof}

\end{CJK}
\end{document}


%%% Local Variables:
%%% mode: latex
%%% TeX-master: t
%%% End:
