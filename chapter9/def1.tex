\documentclass{article}
\usepackage{tikz}
\usepackage{CJKutf8}
\usepackage{amsmath}
\usepackage{amsthm}
\usepackage{ragged2e}
\justifying\let\raggedright\justifying

\begin{document}
\begin{CJK}{UTF8}{gbsn}
  \newtheorem*{Def}{定义}
  \huge
\begin{Def}\justifying\let\raggedright\justifying
图$G$称为被嵌入平面$S$内,如果$G$的图解已画在$S$上,而且任意两条边均不相交(除可能在端点相交外)。已嵌入平面内的图称为{\bfseries 平面图}。如果一个图可以嵌入平面,则称此图为{\bfseries 可平面的}。
\end{Def}

\end{CJK}
\end{document}


%%% Local Variables:
%%% mode: latex
%%% TeX-master: t
%%% End:
