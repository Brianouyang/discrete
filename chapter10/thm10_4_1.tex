\documentclass{article}
\usepackage{tikz}
\usepackage{CJKutf8}
\usepackage{amsmath}
\usepackage{amsthm}
\begin{document}
\begin{CJK}{UTF8}{gbsn}
  \newtheorem*{Thm}{定理}
  \huge
  \begin{Thm}
    设$D=(V,A)$为一个有$p$个顶点$q$条弧的图,$V=\{v_1,v_2,\ldots,v_p\}$,$p\times p$矩阵$B$为$D$的邻接矩阵,则$D$中$v_i$与$v_j$间长为$l$的通道的条数等于$B^l$的第$i$行第$j$列元素的值。
  \end{Thm}
  \begin{proof}[证明]
      用数学归纳法证明,施归纳于$l$。

  当$l=1$时,结论显然成立。

  假设当$l=k$时结论成立,往证当$l=k+1$时结论也成立。由矩阵乘法的计算规则知:
  \[(B^{k+1})_{ij} = (B^{k}B)_{ij} = \sum_{h=1}^p(B^k)_{ih}B_{hj}\]

  由归纳假设,$(B^k)_{ih}$为从顶点$v_i$到顶点$v_h$长度为$k$的通道的条数。

  由从顶点$v_i$到顶点$v_j$长度为$k+1$的通道的条数为从顶点$v_i$到顶点$v_j$长度为
  $k+1$且倒数第二个顶点依次为$v_1$,$v_2$,$\ldots$,$v_p$的通道的条数之和
  知$(B^{k+1})_{ij}$为从顶点$v_i$到顶点$v_j$长度为$k+1$的通道的条数。
  \end{proof}

\end{CJK}
\end{document}


%%% Local Variables:
%%% mode: latex
%%% TeX-master: t
%%% End:
