\documentclass{article}
\usepackage{tikz}
\usepackage{CJKutf8}
\usepackage{amsmath}
\usepackage{amsthm}
\begin{document}
\begin{CJK}{UTF8}{gbsn}
  \newtheorem*{Exercise}{习题}
  \newtheorem*{Def}{定义}
  \Large
    \begin{Def}
    设$X$,$Y$,$Z$为任意三个非空集合。一个 从 $X\times Y$到$Z$的映射 $\phi$ 称 为 $X$与$Y$到$Z$的一个二元(代数)运算。当$X=Y=Z$时,则称$\phi$为$X$上的二元(代数)运算。
  \end{Def}
  \begin{Def}
    从集合$X$到$Y$的任一映射称为从$X$到$Y$的一元(代数)运算。如果$X=Y$,则从$X$到$X$的映射称为$X$上的一元(代数)运算。
  \end{Def}
  \begin{Def}
    设$A_1, A_2, \cdots, A_n, D$为非空集合。一个从 $A_1\times A_2\times \cdots \times A_n$到$D$的映射$\phi$称为$A_1, A_2, \cdots, A_n$到$D$的一个$n$元(代数)运算。
    如果$A_1=A_2=\cdots=A_n=D=A$,则称$\phi$为$A$上的$n$元代数运算。
  \end{Def}

       \begin{Def}
    一个集合及其在该集合上定义的若干个代数运算合称为一个代数系。
  \end{Def}
  \begin{Def}
    设$(S,+)$与$(T, \oplus)$为两个代数系。如果存在一个一一对应$\phi:S\to T$, 使得$\forall x, y \in S$,有
    \begin{align*}
      \phi(x+y) &= \phi(x) \oplus \phi(y),
    \end{align*}
    则称代数系$(S,+)$与$(T, \oplus)$同构,并记为$S\cong T$, $\phi$称为这两个代数系之间的一个同构。
  \end{Def}


\end{CJK}
\end{document}


%%% Local Variables:
%%% mode: latex
%%% TeX-master: t
%%% End:
