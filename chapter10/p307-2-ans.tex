\documentclass{article}
\usepackage{tikz}
\usepackage{CJKutf8}
\usepackage{amsmath}
\usepackage{amsthm}
\begin{document}
\begin{CJK}{UTF8}{gbsn}
  \newtheorem*{Exercise}{习题}
\begin{Exercise}[p307-2]
  设$D$为一个有$p$个顶点$q$条弧的强连通的有向图,则$q$至少是多大?
\end{Exercise}
\begin{proof}[答]
  当$p=1$时,$q=0$;当$p>1$时,$q$至少为$p$。

  当$p>1$时,设$u$和$v$为$D$的两个顶点,由$D$为强连通的知从$u$到$v$有一条有向路$uu_1u_2\ldots u_nv$,从$v$到$u$有一条有向路$vu_{n+2}u_{n+3}\ldots u_{n+m}u$。
  考虑有向闭通道$W=uu_1u_2\ldots u_nvu_{n+2}u_{n+3}\ldots u_{n+m}u$,记$u_0=u$,$v_{n+1}=v$。设$u_j$为$W$上第一个与前面的某个顶点$u_i$重复的顶点,那么$u_iu_{i+1}\ldots u_j$构成了$D$中的一个圈。这证明了当$p>1$时,任意一个强连通图中必定有圈。因此,抹去$D$中所有弧的方向所得到的无向图为连通的,且至少有$1$个圈,去掉该圈上的一条边,所得到的无向图仍然为连通的,从而$q-1\geq p-1$,即$q\geq p$。

  显然由$p$个顶点$v_1,v_2,\ldots,v_p$依次相连所构成的圈$v_1v_2\ldots v_pv_1$有$p$条弧。因此$q$至少为$p$。
 
\end{proof}

\end{CJK}
\end{document}


%%% Local Variables:
%%% mode: latex
%%% TeX-master: t
%%% End:
